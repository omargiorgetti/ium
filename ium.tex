%%%%%%%%%%%%%%%%%%%%%%%%%%%%%%%%%%%%%%%%%%%%%%%%%%%%%%%%%%%%%%%%%%%
%%                                                               %%
%%  LaTeX2e template for the Proceedings of CMMSE            %%
%%                                                               %%
%%  Please don't change this format, which gives the appropiate  %%
%%  appearance for the CMMSE series.                             %%
%%                                                               %%
%%%%%%%%%%%%%%%%%%%%%%%%%%%%%%%%%%%%%%%%%%%%%%%%%%%%%%%%%%%%%%%%%%%

\documentclass{ium}
\usepackage{amssymb}
\usepackage[pdftex]{graphicx}
\usepackage{epstopdf}
\epstopdfsetup{update}

\begin{document}
 
\title{Dashboard per la Business Intelligence}
\author{Omar Giorgetti}{omar.giorgetti@gmail.com}{1}
\affiliation{1}{Corso di Laurea in Informatica}{Univerit\`a agli studi di Firenze}
\markboth{{}}{} 
\markright{\thesection\ {}}
\begin{abstract}

Nell'ultimo decennio, l'aumento spropositato dei dati a disposizione ha contribuito allo sviluppo sempre maggiore di strumenti che consentissero di rappresentare in modo chiaro le informazioni nascoste, migliorandone la conoscenza e l'interpretazione.
Gli strumenti utilizzati a questi scopi sono quelli che rientrano sotto il concetto di \textbf{Business Intelligence} e in modo specifico \textbf{Cruscotti} o \textbf{Dashboard}.

Il mondo dei dati ha avuto inoltre una forte evoluzione dal punto di vista degli strumenti di analisi come Machine Learning e Data Mining e di tutte quelle tecniche che rientrano nel campo delle Data Science. Sviluppo dovuto ad una riduzione dei costi, dei tempi nell'accesso al "calcolo" grazie alla flessibilità dei sistemi di architetture a Container, a tecniche di Bigdata ( una su tutte la piattaforma Hadoop). 

I sistemi di Business Intelligence e le Data Science dall'altra si integrano nel fornire una esperienza finale all'utente sempre più intensa. Nel articolo presente presenteremo alcune tecniche per rendere questa esperienza anche utile.

Nel

\keywords{Business Intelligence, Dashboard, Business Analytics, Big Data, Data Integration, Layout, DataWarehousing, Data Science, Big Data}
\end{abstract}


%
%  Main text of the article.
%


\section{Introduzione}

Lo sviluppo dei sistemi di Business Intelligence degli ultimi dieci anni, grazie anche ad una aumento delle risorse di calcolo dei computer, ha fatto nascere possibilità di business che si sono concretizzate in diverse realtà aziendali. L'enorme quantità di dati a disposizione doveva in un qualche modo esser rappresentata. Da questo punto di vista molte aziende di software hanno sviluppato prodotti proprietari per soddisfare queste esigenze. Di pari passo con questi sviluppi, anche la comunità open source si è organizzata ed ha rilasciato una serie di prodotti che negli ultimi anni stanno rubando fette di mercato importanti. La comunità open source, come del resto anche con altre tipologie di prodotto, affianca alla versione comunity una versione Enterprise con funzionalità aggiuntive e l'assistenza di aziende specializzate.

Allo stesso modo l'evoluzione delle Data Science ha dato un forte spinta verso un insieme di tecniche e di linguaggi di programmazione che forniscono strumenti per raccogliere, analizzare e pubblicare dati ed elaborazioni. Python in notevole sviluppo ma anche software statistici come R che forniscono strumenti veloci di pubblicazione dei risultati con al loro interno un vero e prorpio motore statistico.

I concetti che tratteremo nel seguito dell'articolo riguardano la rappresentazione in forma grafica e tabella di dati e come questi devono essere organizzati. Parlemo di tabelle e grafici, in modo unico, ma anche come questi devono essere messi insieme al fine di costrutire un dashboard.

Questa la definizione in inglese secondo Stephan Few:
\begin{quotation}
"Visual display of the most information needed to achieve one or more objectives which fits entirely on a single computer screen so it can be monitored at a glance"
\end{quotation}



\section{Una breve storia}


\section{Metodi di valutazione utilizzati}
\section{I dashboard sotto analisi}
\begin{center}




	\begin{figure*}
	\includegraphics[scale=1]{../../../Immagini/Dashboard_pentaho.png} 
	\caption{Dashboard Pentaho}

	\end{figure*}
\end{center}

\section{Visual Perception}

La vista è sicuramente il senso che domina qualsiasi altra sensazione per la quantità di informazioni che viene percepita (volume, bandwith,nuance)

Quindi anche il pensiero è fortemente influenzato dalla vista.

Gli occhi e la corteccia sono una dei più potenti elaboratori di una enorme mole di informazioni recepite in modo parallelo e trasmesso ad altissima velocità al centro cognitivo. Processing, Perception e cognition sono strettamente correlati ma presentano comunque dei punti di forte instabilità. Da un certo di punto di vista è scorretto chiamarla instabilità, sarebbe meglio considerare che comunque ci sono delle regole che tutto questo sistema deve seguire. Regole che consentono per esempio di vedere un oggetto in determinate condizioni ma di non vederlo più, a anche se esiste sempre, se queste condizioni cambiano.
Per questo è la percezione l'aspetto da studiare, proprio perchè quando i dati vengono presentati in un dato modo vengono percepiti in modo corretto. Capire come  funziona la percezione, consente di definire un insieme di regole per visualizzare le informazioni. Questo consente anche di poter rappresentare i dati in modo tale da enfatizzare quelli più rilevanti ed informativi.
Logicamente non rispettare queste regole può rendere la rappresentazione incompresibile.

Le rappresentazioni devono quindi essere tale da enfatizzare i dati importanti rispetto al resto. Rappresentare significa anche organizzare le informazioni con l'obiettivo di dare senso alle informazioni diverse, dargli un significato in modo tale che sia percepite.

I principi di Gelstat e i preattintive attribute sono alla base per raggiungere questi obiettivi. Sopratutto cercandone di capire il come e il perchè.

Analizzare la Visual Perception consisete nell'approfondire tre aspetti: 
- I limiti della memoria visuale a breve termine,
- la decodifica visuale ai fini della precezione,
- i principi di Gelstat.


\section{Gelstat}
Vediamo nel dettaglio i principi di Gelstat applicati ai dashboard e in generale alle rappresentazioni grafiche.
\subsection{Legge di superfice}
\subsection{Legge di vicinanza}
\subsection{Legge di forma chiusa}
\subsection{Legge di simmetria}
\subsection{Legge di buona continuazione o continuità di direzione}


\section{}

Kuzica nel suo testo pone bene in enfasi il concetto di percezione parlando di presenza fenomenica. 
L'esistenza, sul piano della realtà percettiva, di una certo aspetto o rapporto (movimento, forma,localizzazione, numero) non si può sempre spiegare facendo riferimento all'esistenza di quell'aspetto o rapporto sul piano della realtaà fisica.
La disposizione di alcuni oggetti può far non solo immaginare ma anche vedere in modo chiaro oggetti che non sono ficicamenti presenti. Kuzica p274
Allo stesso modo anche la trasparenza è uno di quei fenomeni che non esistono nella realtà. Infatti dal punto di vista fisico la parte di sovrapposizione non è altro che un altro oggetto di colore diverso.

La parte fenomenica può anche mancare. Infatti esistono situazioni dove è presente la sola parte fisica e la mente umana è cieca nei confronti dell'oggetto che è presente (Kuzica p29).  
La percezione, o meglio il processo percettivo è tutt'altro che una semplice registrazione passiva dell'ambiente fisico. Esistono situazioni dove la superficialità può trarre in inganno. Queste si verificano quando è presente una discrepanza, tra colore, forma la grandezza e nel rapporto tra la farte fenomenica e fisica. Qui si il contrasto per chiarezza, due oggetti dello stesso colore su sfondi di colori opposti non sembrabo dello stesso colore. Oppure l'illusione di Zoelner, dove bastoncini dentati paralleli sembra non paralleri, secondo le illusioni ottico-geometriche.

\section{Colore}

Colori con lunghezze d'onda diverse devo essere messe a fuoco a distanza diverse, con conseguente rifocalizzazione ed affaticamento.

La saturazione, cioè l'assenza di bianco nel colore. Più è saturo e meno bianco è presente.

Conviene usare colori poco saturi, perchè i colori saturi, cioè puri richiedono una maggiore messa a fuoco. AL limite possono essere utilizzati per evidenziare aspetti

Un modo diverso di interpretare i colori, non solo rgb



\section{La valutazione}
%\newpage

%\section{Figures}

%\section*{Acknowledgements}


%
%  Bibliography. Follow the usual conventions.
%

\begin{thebibliography}{99}

\bibitem{book}
  {\sc Stephen Few},
  {\em Information Dashboard Design},
  O'Really, 2006.

\bibitem{book}
  {\sc Colin Ware},
  {\em Information Visualization: Perception for Design}.

\bibitem{book}
  {\sc Gaetano Kaniza},
  {\em Grammatica del vedere. Saggi su percezione e Gelstat}.


%\bibitem{paper}
%  {\sc E. Witten},
%  {\em Supersymmetry and Morse theory},
%  {\rm J.~Diff.\ Geom.} {\bf 17} (1982) 661--692.

\end{thebibliography}

\end{document}
